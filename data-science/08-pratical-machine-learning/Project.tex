\documentclass[]{article}
\usepackage{lmodern}
\usepackage{amssymb,amsmath}
\usepackage{ifxetex,ifluatex}
\usepackage{fixltx2e} % provides \textsubscript
\ifnum 0\ifxetex 1\fi\ifluatex 1\fi=0 % if pdftex
  \usepackage[T1]{fontenc}
  \usepackage[utf8]{inputenc}
\else % if luatex or xelatex
  \ifxetex
    \usepackage{mathspec}
  \else
    \usepackage{fontspec}
  \fi
  \defaultfontfeatures{Ligatures=TeX,Scale=MatchLowercase}
\fi
% use upquote if available, for straight quotes in verbatim environments
\IfFileExists{upquote.sty}{\usepackage{upquote}}{}
% use microtype if available
\IfFileExists{microtype.sty}{%
\usepackage{microtype}
\UseMicrotypeSet[protrusion]{basicmath} % disable protrusion for tt fonts
}{}
\usepackage[margin=1in]{geometry}
\usepackage{hyperref}
\hypersetup{unicode=true,
            pdftitle={Pratical Machine Learning - Course Project},
            pdfborder={0 0 0},
            breaklinks=true}
\urlstyle{same}  % don't use monospace font for urls
\usepackage{color}
\usepackage{fancyvrb}
\newcommand{\VerbBar}{|}
\newcommand{\VERB}{\Verb[commandchars=\\\{\}]}
\DefineVerbatimEnvironment{Highlighting}{Verbatim}{commandchars=\\\{\}}
% Add ',fontsize=\small' for more characters per line
\usepackage{framed}
\definecolor{shadecolor}{RGB}{248,248,248}
\newenvironment{Shaded}{\begin{snugshade}}{\end{snugshade}}
\newcommand{\KeywordTok}[1]{\textcolor[rgb]{0.13,0.29,0.53}{\textbf{#1}}}
\newcommand{\DataTypeTok}[1]{\textcolor[rgb]{0.13,0.29,0.53}{#1}}
\newcommand{\DecValTok}[1]{\textcolor[rgb]{0.00,0.00,0.81}{#1}}
\newcommand{\BaseNTok}[1]{\textcolor[rgb]{0.00,0.00,0.81}{#1}}
\newcommand{\FloatTok}[1]{\textcolor[rgb]{0.00,0.00,0.81}{#1}}
\newcommand{\ConstantTok}[1]{\textcolor[rgb]{0.00,0.00,0.00}{#1}}
\newcommand{\CharTok}[1]{\textcolor[rgb]{0.31,0.60,0.02}{#1}}
\newcommand{\SpecialCharTok}[1]{\textcolor[rgb]{0.00,0.00,0.00}{#1}}
\newcommand{\StringTok}[1]{\textcolor[rgb]{0.31,0.60,0.02}{#1}}
\newcommand{\VerbatimStringTok}[1]{\textcolor[rgb]{0.31,0.60,0.02}{#1}}
\newcommand{\SpecialStringTok}[1]{\textcolor[rgb]{0.31,0.60,0.02}{#1}}
\newcommand{\ImportTok}[1]{#1}
\newcommand{\CommentTok}[1]{\textcolor[rgb]{0.56,0.35,0.01}{\textit{#1}}}
\newcommand{\DocumentationTok}[1]{\textcolor[rgb]{0.56,0.35,0.01}{\textbf{\textit{#1}}}}
\newcommand{\AnnotationTok}[1]{\textcolor[rgb]{0.56,0.35,0.01}{\textbf{\textit{#1}}}}
\newcommand{\CommentVarTok}[1]{\textcolor[rgb]{0.56,0.35,0.01}{\textbf{\textit{#1}}}}
\newcommand{\OtherTok}[1]{\textcolor[rgb]{0.56,0.35,0.01}{#1}}
\newcommand{\FunctionTok}[1]{\textcolor[rgb]{0.00,0.00,0.00}{#1}}
\newcommand{\VariableTok}[1]{\textcolor[rgb]{0.00,0.00,0.00}{#1}}
\newcommand{\ControlFlowTok}[1]{\textcolor[rgb]{0.13,0.29,0.53}{\textbf{#1}}}
\newcommand{\OperatorTok}[1]{\textcolor[rgb]{0.81,0.36,0.00}{\textbf{#1}}}
\newcommand{\BuiltInTok}[1]{#1}
\newcommand{\ExtensionTok}[1]{#1}
\newcommand{\PreprocessorTok}[1]{\textcolor[rgb]{0.56,0.35,0.01}{\textit{#1}}}
\newcommand{\AttributeTok}[1]{\textcolor[rgb]{0.77,0.63,0.00}{#1}}
\newcommand{\RegionMarkerTok}[1]{#1}
\newcommand{\InformationTok}[1]{\textcolor[rgb]{0.56,0.35,0.01}{\textbf{\textit{#1}}}}
\newcommand{\WarningTok}[1]{\textcolor[rgb]{0.56,0.35,0.01}{\textbf{\textit{#1}}}}
\newcommand{\AlertTok}[1]{\textcolor[rgb]{0.94,0.16,0.16}{#1}}
\newcommand{\ErrorTok}[1]{\textcolor[rgb]{0.64,0.00,0.00}{\textbf{#1}}}
\newcommand{\NormalTok}[1]{#1}
\usepackage{graphicx,grffile}
\makeatletter
\def\maxwidth{\ifdim\Gin@nat@width>\linewidth\linewidth\else\Gin@nat@width\fi}
\def\maxheight{\ifdim\Gin@nat@height>\textheight\textheight\else\Gin@nat@height\fi}
\makeatother
% Scale images if necessary, so that they will not overflow the page
% margins by default, and it is still possible to overwrite the defaults
% using explicit options in \includegraphics[width, height, ...]{}
\setkeys{Gin}{width=\maxwidth,height=\maxheight,keepaspectratio}
\IfFileExists{parskip.sty}{%
\usepackage{parskip}
}{% else
\setlength{\parindent}{0pt}
\setlength{\parskip}{6pt plus 2pt minus 1pt}
}
\setlength{\emergencystretch}{3em}  % prevent overfull lines
\providecommand{\tightlist}{%
  \setlength{\itemsep}{0pt}\setlength{\parskip}{0pt}}
\setcounter{secnumdepth}{0}
% Redefines (sub)paragraphs to behave more like sections
\ifx\paragraph\undefined\else
\let\oldparagraph\paragraph
\renewcommand{\paragraph}[1]{\oldparagraph{#1}\mbox{}}
\fi
\ifx\subparagraph\undefined\else
\let\oldsubparagraph\subparagraph
\renewcommand{\subparagraph}[1]{\oldsubparagraph{#1}\mbox{}}
\fi

%%% Use protect on footnotes to avoid problems with footnotes in titles
\let\rmarkdownfootnote\footnote%
\def\footnote{\protect\rmarkdownfootnote}

%%% Change title format to be more compact
\usepackage{titling}

% Create subtitle command for use in maketitle
\newcommand{\subtitle}[1]{
  \posttitle{
    \begin{center}\large#1\end{center}
    }
}

\setlength{\droptitle}{-2em}

  \title{Pratical Machine Learning - Course Project}
    \pretitle{\vspace{\droptitle}\centering\huge}
  \posttitle{\par}
    \author{}
    \preauthor{}\postauthor{}
    \date{}
    \predate{}\postdate{}
  
\usepackage{booktabs}
\usepackage{longtable}
\usepackage{array}
\usepackage{multirow}
\usepackage{wrapfig}
\usepackage{float}
\usepackage{colortbl}
\usepackage{pdflscape}
\usepackage{tabu}
\usepackage{threeparttable}
\usepackage{threeparttablex}
\usepackage[normalem]{ulem}
\usepackage{makecell}
\usepackage{xcolor}

\begin{document}
\maketitle

\subsection{Introduction}\label{introduction}

Using devices such as Jawbone Up, Nike FuelBand, and Fitbit it is now
possible to collect a large amount of data about personal activity. One
thing that people regularly do is quantify how much of a particular
activity they do, but they rarely quantify how well they do it.

In this
\href{https://www.coursera.org/learn/practical-machine-learning/supplement/PvInj/course-project-instructions-read-first}{project},
the goal is creating a model to predict the manner in which they did the
exercise. To do so, we are going to use the data from the study about
\href{http://groupware.les.inf.puc-rio.br/public/papers/2013.Velloso.QAR-WLE.pdf}{Qualitative
Activity Recognition of Weight Lifting Exercises} {[}1{]} that register
the accelerometers on the belt, forearm, arm, and dumbell of 6
participants. In this study, the participants were asked to perform
barbell lifts correctly and incorrectly in 5 different ways. More
information about this data is available on and more details about it
can be access by this website
\url{http://groupware.les.inf.puc-rio.br/har}, in the section on the
Weight Lifting Exercise Dataset.

\subsection{Assignment}\label{assignment}

As said previously, the goal of your project is to predict the manner in
which they did the exercise. This is the ``classe'' variable in the
training set. To do so, it is allowed to use any of the other variables
to predict with. This paper must report how built the created model, how
were used the cross validation, what is the expected out of sample
error, and the cause of the choices maded. After that, it must present
the prediction result of the model over 20 different test cases.

\begin{Shaded}
\begin{Highlighting}[]
\KeywordTok{options}\NormalTok{(}\DataTypeTok{scipen=}\DecValTok{999}\NormalTok{)         }\CommentTok{# make the number printer more readable}
\KeywordTok{Sys.setenv}\NormalTok{(}\DataTypeTok{LANG =} \StringTok{"en"}\NormalTok{)     }\CommentTok{# show messages on english}
\KeywordTok{Sys.setenv}\NormalTok{(}\DataTypeTok{LANGUAGE =} \StringTok{"en"}\NormalTok{) }\CommentTok{# show messages on english}
\KeywordTok{rm}\NormalTok{(}\DataTypeTok{list=}\KeywordTok{ls}\NormalTok{())               }\CommentTok{# remove other data from env, if any}
\KeywordTok{set.seed}\NormalTok{(}\DecValTok{123}\NormalTok{)               }\CommentTok{# set a seed to ensure get always the same results}
\end{Highlighting}
\end{Shaded}

\subsection{Data}\label{data}

\begin{Shaded}
\begin{Highlighting}[]
\NormalTok{trainDataLink <-}\StringTok{ 'https://d396qusza40orc.cloudfront.net/predmachlearn/pml-training.csv'}
\NormalTok{trainDataFile <-}\StringTok{ 'pml-training.csv'}\NormalTok{;}
\NormalTok{testDataLink  <-}\StringTok{ 'https://d396qusza40orc.cloudfront.net/predmachlearn/pml-testing.csv'}
\NormalTok{testDataFile  <-}\StringTok{ 'pml-testing'}\NormalTok{;}
\end{Highlighting}
\end{Shaded}

The data for this assignment is divided among
\href{https://d396qusza40orc.cloudfront.net/predmachlearn/pml-training.csv}{trainng
data} and
\href{https://d396qusza40orc.cloudfront.net/predmachlearn/pml-testing.csv}{test
data}:

\subsection{Loading Libraries}\label{loading-libraries}

\begin{Shaded}
\begin{Highlighting}[]
\CommentTok{# loading required libraries}
\CommentTok{# install.packages(c("devtools", "knitr","ggplot2","data.table","caret",}
\CommentTok{#   "doParallel","e1071","rpart","rpart.plot","rattle","gridExtra","kableExtra", "ggRandomForests"), dependencies = TRUE)}
\KeywordTok{library}\NormalTok{(}\StringTok{'knitr'}\NormalTok{)}
\KeywordTok{library}\NormalTok{(}\StringTok{'ggplot2'}\NormalTok{)}
\KeywordTok{library}\NormalTok{(}\StringTok{'data.table'}\NormalTok{)}
\KeywordTok{library}\NormalTok{(}\StringTok{'caret'}\NormalTok{)}
\end{Highlighting}
\end{Shaded}

\begin{verbatim}
## Loading required package: lattice
\end{verbatim}

\begin{Shaded}
\begin{Highlighting}[]
\KeywordTok{library}\NormalTok{(}\StringTok{'devtools'}\NormalTok{)}
\KeywordTok{library}\NormalTok{(}\StringTok{'doParallel'}\NormalTok{) }
\end{Highlighting}
\end{Shaded}

\begin{verbatim}
## Loading required package: foreach
\end{verbatim}

\begin{verbatim}
## Loading required package: iterators
\end{verbatim}

\begin{verbatim}
## Loading required package: parallel
\end{verbatim}

\begin{Shaded}
\begin{Highlighting}[]
\KeywordTok{library}\NormalTok{(}\StringTok{'e1071'}\NormalTok{)}
\KeywordTok{library}\NormalTok{(}\StringTok{'rattle'}\NormalTok{)}
\end{Highlighting}
\end{Shaded}

\begin{verbatim}
## Rattle: A free graphical interface for data science with R.
## Version 5.2.0 Copyright (c) 2006-2018 Togaware Pty Ltd.
## Type 'rattle()' to shake, rattle, and roll your data.
\end{verbatim}

\begin{Shaded}
\begin{Highlighting}[]
\KeywordTok{library}\NormalTok{(}\StringTok{'rpart'}\NormalTok{)}
\KeywordTok{library}\NormalTok{(}\StringTok{'rpart.plot'}\NormalTok{)}
\KeywordTok{library}\NormalTok{(}\StringTok{'gridExtra'}\NormalTok{)}
\KeywordTok{library}\NormalTok{(}\StringTok{'kableExtra'}\NormalTok{)}
\KeywordTok{library}\NormalTok{(}\StringTok{'ggRandomForests'}\NormalTok{)}
\end{Highlighting}
\end{Shaded}

\begin{verbatim}
## Loading required package: randomForestSRC
\end{verbatim}

\begin{verbatim}
## 
##  randomForestSRC 2.8.0 
##  
##  Type rfsrc.news() to see new features, changes, and bug fixes. 
## 
\end{verbatim}

\begin{verbatim}
## 
## Attaching package: 'randomForestSRC'
\end{verbatim}

\begin{verbatim}
## The following objects are masked from 'package:e1071':
## 
##     impute, tune
\end{verbatim}

\begin{verbatim}
## 
## Attaching package: 'ggRandomForests'
\end{verbatim}

\begin{verbatim}
## The following object is masked from 'package:randomForestSRC':
## 
##     partial.rfsrc
\end{verbatim}

\begin{Shaded}
\begin{Highlighting}[]
\KeywordTok{registerDoParallel}\NormalTok{(}\DataTypeTok{cores=}\DecValTok{4}\NormalTok{)}
\end{Highlighting}
\end{Shaded}

\subsection{Loading and preprocessing the
data}\label{loading-and-preprocessing-the-data}

Download the data and load it into data.table.

\begin{Shaded}
\begin{Highlighting}[]
\ControlFlowTok{if}\NormalTok{ ( }\OperatorTok{!}\KeywordTok{file.exists}\NormalTok{(trainDataFile) ) \{}
  \KeywordTok{download.file}\NormalTok{(trainDataLink, trainDataFile)}
\NormalTok{\}}
\ControlFlowTok{if}\NormalTok{ ( }\OperatorTok{!}\KeywordTok{file.exists}\NormalTok{(testDataFile) ) \{}
  \KeywordTok{download.file}\NormalTok{(testDataLink, testDataFile)}
\NormalTok{\}}
\NormalTok{trainData <-}\StringTok{ }\KeywordTok{read.csv}\NormalTok{( trainDataFile, }\DataTypeTok{na.strings=}\KeywordTok{c}\NormalTok{(}\StringTok{'#DIV/0!'}\NormalTok{, }\StringTok{''}\NormalTok{, }\StringTok{'NA'}\NormalTok{), }\DataTypeTok{stringsAsFactors =} \OtherTok{FALSE}\NormalTok{)}
\NormalTok{testData  <-}\StringTok{ }\KeywordTok{read.csv}\NormalTok{( testDataFile,  }\DataTypeTok{na.strings=}\KeywordTok{c}\NormalTok{(}\StringTok{'#DIV/0!'}\NormalTok{, }\StringTok{''}\NormalTok{, }\StringTok{'NA'}\NormalTok{), }\DataTypeTok{stringsAsFactors =} \OtherTok{FALSE}\NormalTok{)}
\end{Highlighting}
\end{Shaded}

\subsubsection{Removing Null Columns}\label{removing-null-columns}

Remove all columns with 97\% or more rows with null values.

\begin{Shaded}
\begin{Highlighting}[]
\NormalTok{MAX_PERCENT_OF_NA_VALUES =}\StringTok{ }\FloatTok{0.97}
\NormalTok{fields <-}\StringTok{ }\KeywordTok{names}\NormalTok{(trainData)}
\NormalTok{size <-}\KeywordTok{nrow}\NormalTok{(trainData)}
\NormalTok{fieldsToRemove <-}\StringTok{ }\KeywordTok{c}\NormalTok{()}

\ControlFlowTok{for}\NormalTok{(field }\ControlFlowTok{in}\NormalTok{ fields) \{}
\NormalTok{  column <-}\StringTok{ }\NormalTok{trainData[[field]]}
\NormalTok{  percentOfNA <-}\StringTok{ }\NormalTok{( (}\KeywordTok{length}\NormalTok{(column[}\KeywordTok{is.na}\NormalTok{(column)])) }\OperatorTok{/}\StringTok{ }\NormalTok{size )}
  \ControlFlowTok{if}\NormalTok{( percentOfNA }\OperatorTok{>=}\StringTok{ }\NormalTok{MAX_PERCENT_OF_NA_VALUES ) \{}
\NormalTok{    fieldsToRemove[}\KeywordTok{length}\NormalTok{(fieldsToRemove)}\OperatorTok{+}\DecValTok{1}\NormalTok{] <-}\StringTok{ }\NormalTok{field;}
\NormalTok{  \}}
\NormalTok{\}}
\KeywordTok{print}\NormalTok{(}\KeywordTok{paste}\NormalTok{(}\StringTok{"removing these fields for having to many empty values: ("}\NormalTok{,}\KeywordTok{paste}\NormalTok{(fieldsToRemove, }\DataTypeTok{collapse =} \StringTok{', '}\NormalTok{),}\StringTok{")"}\NormalTok{));}
\end{Highlighting}
\end{Shaded}

\begin{verbatim}
## [1] "removing these fields for having to many empty values: ( kurtosis_roll_belt, kurtosis_picth_belt, kurtosis_yaw_belt, skewness_roll_belt, skewness_roll_belt.1, skewness_yaw_belt, max_roll_belt, max_picth_belt, max_yaw_belt, min_roll_belt, min_pitch_belt, min_yaw_belt, amplitude_roll_belt, amplitude_pitch_belt, amplitude_yaw_belt, var_total_accel_belt, avg_roll_belt, stddev_roll_belt, var_roll_belt, avg_pitch_belt, stddev_pitch_belt, var_pitch_belt, avg_yaw_belt, stddev_yaw_belt, var_yaw_belt, var_accel_arm, avg_roll_arm, stddev_roll_arm, var_roll_arm, avg_pitch_arm, stddev_pitch_arm, var_pitch_arm, avg_yaw_arm, stddev_yaw_arm, var_yaw_arm, kurtosis_roll_arm, kurtosis_picth_arm, kurtosis_yaw_arm, skewness_roll_arm, skewness_pitch_arm, skewness_yaw_arm, max_roll_arm, max_picth_arm, max_yaw_arm, min_roll_arm, min_pitch_arm, min_yaw_arm, amplitude_roll_arm, amplitude_pitch_arm, amplitude_yaw_arm, kurtosis_roll_dumbbell, kurtosis_picth_dumbbell, kurtosis_yaw_dumbbell, skewness_roll_dumbbell, skewness_pitch_dumbbell, skewness_yaw_dumbbell, max_roll_dumbbell, max_picth_dumbbell, max_yaw_dumbbell, min_roll_dumbbell, min_pitch_dumbbell, min_yaw_dumbbell, amplitude_roll_dumbbell, amplitude_pitch_dumbbell, amplitude_yaw_dumbbell, var_accel_dumbbell, avg_roll_dumbbell, stddev_roll_dumbbell, var_roll_dumbbell, avg_pitch_dumbbell, stddev_pitch_dumbbell, var_pitch_dumbbell, avg_yaw_dumbbell, stddev_yaw_dumbbell, var_yaw_dumbbell, kurtosis_roll_forearm, kurtosis_picth_forearm, kurtosis_yaw_forearm, skewness_roll_forearm, skewness_pitch_forearm, skewness_yaw_forearm, max_roll_forearm, max_picth_forearm, max_yaw_forearm, min_roll_forearm, min_pitch_forearm, min_yaw_forearm, amplitude_roll_forearm, amplitude_pitch_forearm, amplitude_yaw_forearm, var_accel_forearm, avg_roll_forearm, stddev_roll_forearm, var_roll_forearm, avg_pitch_forearm, stddev_pitch_forearm, var_pitch_forearm, avg_yaw_forearm, stddev_yaw_forearm, var_yaw_forearm )"
\end{verbatim}

\begin{Shaded}
\begin{Highlighting}[]
\CommentTok{# using the same field list to train and test to avoid different field list}
\NormalTok{trainData <-}\StringTok{ }\NormalTok{trainData[ , }\OperatorTok{!}\NormalTok{(}\KeywordTok{colnames}\NormalTok{(trainData) }\OperatorTok\StringTok{ }\NormalTok{fieldsToRemove)]}
\NormalTok{testData  <-}\StringTok{ }\NormalTok{testData[  , }\OperatorTok{!}\NormalTok{(}\KeywordTok{colnames}\NormalTok{(testData)  }\OperatorTok\StringTok{ }\NormalTok{fieldsToRemove)]}
\end{Highlighting}
\end{Shaded}

\subsubsection{Replacing Null values by the
Median}\label{replacing-null-values-by-the-median}

Replace remaing null values by the median or the most common value.

\begin{Shaded}
\begin{Highlighting}[]
\NormalTok{fields <-}\StringTok{ }\KeywordTok{names}\NormalTok{(trainData)}
\NormalTok{commonValues <-}\StringTok{ }\KeywordTok{c}\NormalTok{()}
\ControlFlowTok{for}\NormalTok{(field }\ControlFlowTok{in}\NormalTok{ fields) \{}
\NormalTok{  column <-}\StringTok{ }\NormalTok{trainData[[field]]}
  \ControlFlowTok{if}\NormalTok{(}\OperatorTok{!}\KeywordTok{is.character}\NormalTok{(column))\{}
\NormalTok{    new_value <-}\StringTok{ }\KeywordTok{median}\NormalTok{(column)}
\NormalTok{  \} }\ControlFlowTok{else}\NormalTok{ \{}
\NormalTok{    new_value <-}\StringTok{ }\KeywordTok{names}\NormalTok{(}\KeywordTok{sort}\NormalTok{(}\KeywordTok{table}\NormalTok{(column),}\DataTypeTok{decreasing=}\OtherTok{TRUE}\NormalTok{)[}\DecValTok{1}\NormalTok{]) }
\NormalTok{  \}}
\NormalTok{  commonValues[field] <-}\StringTok{ }\NormalTok{new_value;}
\NormalTok{\}}
\NormalTok{replaceNullByCommonValues <-}\StringTok{ }\ControlFlowTok{function}\NormalTok{(dataframe,commonValues) \{}
  \ControlFlowTok{for}\NormalTok{(field }\ControlFlowTok{in}\NormalTok{ fields) \{}
\NormalTok{    column <-}\StringTok{ }\NormalTok{dataframe[[field]]}
\NormalTok{    totalNull <-}\StringTok{ }\KeywordTok{length}\NormalTok{(column[}\KeywordTok{is.na}\NormalTok{(column)])}
    \ControlFlowTok{if}\NormalTok{(totalNull }\OperatorTok{>}\StringTok{ }\DecValTok{0}\NormalTok{) \{}
\NormalTok{      commonValue <-}\StringTok{ }\NormalTok{commonValues[field]}
      \KeywordTok{print}\NormalTok{(}\KeywordTok{paste}\NormalTok{(}\StringTok{"replacing"}\NormalTok{,totalNull,}\StringTok{" null values on"}\NormalTok{,field,}\StringTok{"by"}\NormalTok{,commonValue))}
\NormalTok{      column[}\KeywordTok{is.na}\NormalTok{(column)] <-}\StringTok{ }\NormalTok{commonValue}
\NormalTok{      dataframe[[field]] <-}\StringTok{ }\NormalTok{column}
\NormalTok{    \}}
\NormalTok{  \}}
  \KeywordTok{return}\NormalTok{(dataframe)}
\NormalTok{\}}
\NormalTok{trainData <-}\StringTok{ }\KeywordTok{replaceNullByCommonValues}\NormalTok{(trainData,commonValues)}
\NormalTok{testData  <-}\StringTok{ }\KeywordTok{replaceNullByCommonValues}\NormalTok{(testData,commonValues)}
\KeywordTok{print}\NormalTok{(}\StringTok{"all the null values where replaced by the common values"}\NormalTok{)}
\end{Highlighting}
\end{Shaded}

\begin{verbatim}
## [1] "all the null values where replaced by the common values"
\end{verbatim}

\subsection{Remove Near Zero Variance
Columns}\label{remove-near-zero-variance-columns}

\begin{Shaded}
\begin{Highlighting}[]
\NormalTok{nearZeroVarFields <-}\StringTok{ }\KeywordTok{nearZeroVar}\NormalTok{(trainData, }\DataTypeTok{names =} \OtherTok{TRUE}\NormalTok{)}
\NormalTok{trainData <-}\StringTok{ }\NormalTok{trainData[ , }\OperatorTok{!}\NormalTok{(}\KeywordTok{colnames}\NormalTok{(trainData) }\OperatorTok\StringTok{ }\NormalTok{nearZeroVarFields)]}
\ControlFlowTok{if}\NormalTok{( }\KeywordTok{length}\NormalTok{(nearZeroVarFields) }\OperatorTok{>}\StringTok{ }\DecValTok{0}\NormalTok{ ) \{}
  \KeywordTok{cat}\NormalTok{(}\KeywordTok{paste}\NormalTok{(}\StringTok{"removing these fields for having near zero variance: ("}\NormalTok{,}\KeywordTok{paste}\NormalTok{(nearZeroVarFields, }\DataTypeTok{collapse =} \StringTok{', '}\NormalTok{),}\StringTok{")"}\NormalTok{));  }
\NormalTok{\} }\ControlFlowTok{else}\NormalTok{ \{}
  \KeywordTok{cat}\NormalTok{(}\StringTok{"all fields have a acceptable variance"}\NormalTok{)}
\NormalTok{\}}
\end{Highlighting}
\end{Shaded}

\begin{verbatim}
## removing these fields for having near zero variance: ( new_window )
\end{verbatim}

\subsection{Remove Id and Time
columns}\label{remove-id-and-time-columns}

The goal is detect if the exercise is being done correctly or not based
on the detected device data. In this goal, when the data was collected
or who is the user should not affect the result.

\begin{Shaded}
\begin{Highlighting}[]
\NormalTok{fieldsToRemove <-}\StringTok{ }\KeywordTok{c}\NormalTok{(}\StringTok{"X"}\NormalTok{, }\StringTok{"user_name"}\NormalTok{, }\StringTok{"raw_timestamp_part_1"}\NormalTok{, }\StringTok{"raw_timestamp_part_2"}\NormalTok{, }\StringTok{"cvtd_timestamp"}\NormalTok{, }\StringTok{"new_window"}\NormalTok{, }\StringTok{"num_window"}\NormalTok{)}
\NormalTok{trainData <-}\StringTok{ }\NormalTok{trainData[ , }\OperatorTok{!}\NormalTok{(}\KeywordTok{colnames}\NormalTok{(trainData) }\OperatorTok\StringTok{ }\NormalTok{fieldsToRemove)]}
\KeywordTok{cat}\NormalTok{(}\StringTok{"id columns removed"}\NormalTok{)}
\end{Highlighting}
\end{Shaded}

\begin{verbatim}
## id columns removed
\end{verbatim}

\subsection{Separate Validation Data}\label{separate-validation-data}

The train data is separated among the train dataset that will be used to
train the models and the validation dataset that will be used to
validate the accuracy of the created model. In this project the train
dataset will use 90\% of the rows of the train data and the remain 10\%
will be used on the validation dataset.

\begin{Shaded}
\begin{Highlighting}[]
\KeywordTok{set.seed}\NormalTok{(}\DecValTok{123}\NormalTok{)}
\NormalTok{targetFields <-}\StringTok{ }\KeywordTok{c}\NormalTok{(}\StringTok{"classe"}\NormalTok{)}

\NormalTok{trainDataRows               <-}\StringTok{ }\KeywordTok{createDataPartition}\NormalTok{(}\DataTypeTok{y =}\NormalTok{ trainData}\OperatorTok{$}\NormalTok{classe, }\DataTypeTok{p =} \FloatTok{0.9}\NormalTok{, }\DataTypeTok{list =} \OtherTok{FALSE}\NormalTok{)}

\NormalTok{trainDataset                <-}\StringTok{ }\NormalTok{trainData[ trainDataRows,]}
\NormalTok{trainDatasetPredictors      <-}\StringTok{ }\NormalTok{trainData[ trainDataRows, }\OperatorTok{!}\NormalTok{(}\KeywordTok{colnames}\NormalTok{(trainData) }\OperatorTok\StringTok{ }\NormalTok{targetFields)]}
\NormalTok{trainDatasetTarget          <-}\StringTok{ }\KeywordTok{factor}\NormalTok{(trainData[ trainDataRows, ]}\OperatorTok{$}\NormalTok{classe)}

\NormalTok{validationDataset           <-}\StringTok{ }\NormalTok{trainData[}\OperatorTok{-}\NormalTok{trainDataRows,]}
\NormalTok{validationDatasetPredictors <-}\StringTok{ }\NormalTok{trainData[}\OperatorTok{-}\NormalTok{trainDataRows, }\OperatorTok{!}\NormalTok{(}\KeywordTok{colnames}\NormalTok{(trainData) }\OperatorTok\StringTok{ }\NormalTok{targetFields)]}
\NormalTok{validationDatasetTarget     <-}\StringTok{ }\KeywordTok{factor}\NormalTok{(trainData[}\OperatorTok{-}\NormalTok{trainDataRows, ]}\OperatorTok{$}\NormalTok{classe)}

\KeywordTok{cat}\NormalTok{(}\KeywordTok{paste}\NormalTok{(}\StringTok{"train dataset have "}\NormalTok{,}\KeywordTok{nrow}\NormalTok{(trainDataset),}\StringTok{"rows"}\NormalTok{),}\StringTok{"}\CharTok{\textbackslash{}n}\StringTok{"}\NormalTok{)}
\end{Highlighting}
\end{Shaded}

\begin{verbatim}
## train dataset have  17662 rows
\end{verbatim}

\begin{Shaded}
\begin{Highlighting}[]
\KeywordTok{cat}\NormalTok{(}\KeywordTok{paste}\NormalTok{(}\StringTok{"validation dataset have "}\NormalTok{,}\KeywordTok{nrow}\NormalTok{(validationDataset),}\StringTok{"rows"}\NormalTok{),}\StringTok{"}\CharTok{\textbackslash{}n}\StringTok{"}\NormalTok{)}
\end{Highlighting}
\end{Shaded}

\begin{verbatim}
## validation dataset have  1960 rows
\end{verbatim}

\begin{Shaded}
\begin{Highlighting}[]
\KeywordTok{cat}\NormalTok{(}\KeywordTok{paste}\NormalTok{(}\StringTok{"test dataset have "}\NormalTok{,}\KeywordTok{nrow}\NormalTok{(testData),}\StringTok{"rows"}\NormalTok{),}\StringTok{"}\CharTok{\textbackslash{}n}\StringTok{"}\NormalTok{)}
\end{Highlighting}
\end{Shaded}

\begin{verbatim}
## test dataset have  20 rows
\end{verbatim}

\subsection{Creating Models}\label{creating-models}

\begin{Shaded}
\begin{Highlighting}[]
\KeywordTok{set.seed}\NormalTok{(}\DecValTok{123}\NormalTok{)}
\NormalTok{folds =}\StringTok{ }\DecValTok{3}
\NormalTok{modelTrainControl <-}\StringTok{ }\KeywordTok{trainControl}\NormalTok{(}
  \DataTypeTok{method =} \StringTok{"cv"}\NormalTok{,             }\CommentTok{# for “cross-validation”}
  \DataTypeTok{number =}\NormalTok{ folds,            }\CommentTok{# number of k-folds}
  \DataTypeTok{returnResamp =} \StringTok{'final'}\NormalTok{,}
  \DataTypeTok{classProb =} \OtherTok{TRUE}\NormalTok{,}
  \DataTypeTok{returnData =} \OtherTok{FALSE}\NormalTok{,}
  \DataTypeTok{savePredictions =} \OtherTok{FALSE}\NormalTok{,}
  \DataTypeTok{verboseIter =} \OtherTok{TRUE}\NormalTok{,}
  \DataTypeTok{allowParallel =} \OtherTok{TRUE}\NormalTok{,}
  \DataTypeTok{index=}\KeywordTok{createFolds}\NormalTok{(trainDataset}\OperatorTok{$}\NormalTok{classe,}\DataTypeTok{k=}\NormalTok{folds)}
\NormalTok{)}

\NormalTok{preProcess=}\KeywordTok{c}\NormalTok{(}\StringTok{"pca"}\NormalTok{,}\StringTok{"center"}\NormalTok{,}\StringTok{"scale"}\NormalTok{)}
\NormalTok{modelFitBag                  <-}\StringTok{ }\KeywordTok{train}\NormalTok{(classe }\OperatorTok{~}\StringTok{ }\NormalTok{., }\DataTypeTok{data =}\NormalTok{ trainDataset, }\DataTypeTok{method =} \StringTok{"treebag"}\NormalTok{, }\DataTypeTok{preProcess =}\NormalTok{ preProcess, }\DataTypeTok{trControl=}\NormalTok{modelTrainControl)}
\end{Highlighting}
\end{Shaded}

\begin{verbatim}
## Aggregating results
## Fitting final model on full training set
\end{verbatim}

\begin{Shaded}
\begin{Highlighting}[]
\NormalTok{modelKNearestNeighbor        <-}\StringTok{ }\KeywordTok{train}\NormalTok{(classe }\OperatorTok{~}\StringTok{ }\NormalTok{., }\DataTypeTok{data =}\NormalTok{ trainDataset, }\DataTypeTok{method =} \StringTok{"knn"}\NormalTok{,     }\DataTypeTok{preProcess =}\NormalTok{ preProcess, }\DataTypeTok{trControl=}\NormalTok{modelTrainControl)}
\end{Highlighting}
\end{Shaded}

\begin{verbatim}
## Aggregating results
## Selecting tuning parameters
## Fitting k = 5 on full training set
\end{verbatim}

\begin{Shaded}
\begin{Highlighting}[]
\NormalTok{modelRecursivePartition      <-}\StringTok{ }\KeywordTok{train}\NormalTok{(classe }\OperatorTok{~}\StringTok{ }\NormalTok{., }\DataTypeTok{data =}\NormalTok{ trainDataset, }\DataTypeTok{method =} \StringTok{"rpart"}\NormalTok{,   }\DataTypeTok{preProcess =}\NormalTok{ preProcess, }\DataTypeTok{trControl=}\NormalTok{modelTrainControl)}
\end{Highlighting}
\end{Shaded}

\begin{verbatim}
## Aggregating results
## Selecting tuning parameters
## Fitting cp = 0.0331 on full training set
\end{verbatim}

\begin{Shaded}
\begin{Highlighting}[]
\NormalTok{modelGradientBoostingMachine <-}\StringTok{ }\KeywordTok{train}\NormalTok{(classe }\OperatorTok{~}\StringTok{ }\NormalTok{., }\DataTypeTok{data =}\NormalTok{ trainDataset, }\DataTypeTok{method =} \StringTok{"gbm"}\NormalTok{,     }\DataTypeTok{preProcess =}\NormalTok{ preProcess, }\DataTypeTok{trControl=}\NormalTok{modelTrainControl)}
\end{Highlighting}
\end{Shaded}

\begin{verbatim}
## Aggregating results
## Selecting tuning parameters
## Fitting n.trees = 150, interaction.depth = 3, shrinkage = 0.1, n.minobsinnode = 10 on full training set
## Iter   TrainDeviance   ValidDeviance   StepSize   Improve
##      1        1.6094             nan     0.1000    0.1276
##      2        1.5310             nan     0.1000    0.0926
##      3        1.4727             nan     0.1000    0.0735
##      4        1.4271             nan     0.1000    0.0581
##      5        1.3906             nan     0.1000    0.0502
##      6        1.3584             nan     0.1000    0.0460
##      7        1.3299             nan     0.1000    0.0435
##      8        1.3014             nan     0.1000    0.0353
##      9        1.2781             nan     0.1000    0.0338
##     10        1.2560             nan     0.1000    0.0269
##     20        1.1055             nan     0.1000    0.0186
##     40        0.9337             nan     0.1000    0.0088
##     60        0.8294             nan     0.1000    0.0055
##     80        0.7509             nan     0.1000    0.0047
##    100        0.6883             nan     0.1000    0.0036
##    120        0.6343             nan     0.1000    0.0023
##    140        0.5882             nan     0.1000    0.0021
##    150        0.5692             nan     0.1000    0.0015
\end{verbatim}

\begin{Shaded}
\begin{Highlighting}[]
\NormalTok{modelRandomForest            <-}\StringTok{ }\KeywordTok{train}\NormalTok{(classe }\OperatorTok{~}\StringTok{ }\NormalTok{., }\DataTypeTok{data =}\NormalTok{ trainDataset, }\DataTypeTok{method =} \StringTok{"rf"}\NormalTok{,      }\DataTypeTok{preProcess =}\NormalTok{ preProcess, }\DataTypeTok{trControl=}\NormalTok{modelTrainControl)}
\end{Highlighting}
\end{Shaded}

\begin{verbatim}
## Aggregating results
## Selecting tuning parameters
## Fitting mtry = 2 on full training set
\end{verbatim}

\begin{Shaded}
\begin{Highlighting}[]
\NormalTok{allModels <-}\StringTok{ }\KeywordTok{list}\NormalTok{(}
\NormalTok{  modelFitBag, }
\NormalTok{  modelKNearestNeighbor, }
\NormalTok{  modelRecursivePartition, }
\NormalTok{  modelGradientBoostingMachine, }
\NormalTok{  modelRandomForest}
\NormalTok{)}
\KeywordTok{names}\NormalTok{(allModels) <-}\StringTok{ }\KeywordTok{sapply}\NormalTok{(allModels, }\ControlFlowTok{function}\NormalTok{(x) x}\OperatorTok{$}\NormalTok{method)}
\KeywordTok{sort}\NormalTok{(}\KeywordTok{sapply}\NormalTok{(allModels, }\ControlFlowTok{function}\NormalTok{(x) x}\OperatorTok{$}\NormalTok{results}\OperatorTok{$}\NormalTok{Accuracy[}\KeywordTok{length}\NormalTok{(x}\OperatorTok{$}\NormalTok{results}\OperatorTok{$}\NormalTok{Accuracy)]),}\DataTypeTok{decreasing =} \OtherTok{TRUE}\NormalTok{)}
\end{Highlighting}
\end{Shaded}

\begin{verbatim}
##        rf   treebag       knn       gbm     rpart 
## 0.9290286 0.9141377 0.8611988 0.8056844 0.2843393
\end{verbatim}

\begin{Shaded}
\begin{Highlighting}[]
\NormalTok{summaryModels <-}\StringTok{ }\KeywordTok{resamples}\NormalTok{(}
  \KeywordTok{list}\NormalTok{(}
    \DataTypeTok{fitBag=}\NormalTok{modelFitBag,}
    \DataTypeTok{knm=}\NormalTok{modelKNearestNeighbor, }
    \DataTypeTok{rpart=}\NormalTok{modelRecursivePartition, }
    \DataTypeTok{gbm=}\NormalTok{modelGradientBoostingMachine,}
    \DataTypeTok{rf=}\NormalTok{modelRandomForest}
\NormalTok{  )}
\NormalTok{)}
\KeywordTok{summary}\NormalTok{(summaryModels)}
\end{Highlighting}
\end{Shaded}

\begin{verbatim}
## 
## Call:
## summary.resamples(object = summaryModels)
## 
## Models: fitBag, knm, rpart, gbm, rf 
## Number of resamples: 3 
## 
## Accuracy 
##             Min.   1st Qu.    Median      Mean   3rd Qu.      Max. NA's
## fitBag 0.9061491 0.9098511 0.9135530 0.9141377 0.9181320 0.9227111    0
## knm    0.8838967 0.8928655 0.9018342 0.8958778 0.9018684 0.9019025    0
## rpart  0.3357398 0.3670886 0.3984375 0.3784101 0.3997453 0.4010532    0
## gbm    0.7980296 0.8029295 0.8078295 0.8056844 0.8095118 0.8111942    0
## rf     0.9327331 0.9391077 0.9454823 0.9421355 0.9468366 0.9481909    0
## 
## Kappa 
##             Min.   1st Qu.    Median      Mean   3rd Qu.      Max. NA's
## fitBag 0.8812924 0.8859546 0.8906168 0.8913622 0.8963971 0.9021774    0
## knm    0.8531573 0.8644671 0.8757769 0.8682613 0.8758133 0.8758498    0
## rpart  0.1415161 0.1772751 0.2130341 0.1907712 0.2153988 0.2177635    0
## gbm    0.7441349 0.7502867 0.7564386 0.7537984 0.7586302 0.7608218    0
## rf     0.9148974 0.9229492 0.9310009 0.9267757 0.9327148 0.9344288    0
\end{verbatim}

\begin{Shaded}
\begin{Highlighting}[]
\KeywordTok{bwplot}\NormalTok{(summaryModels)}
\end{Highlighting}
\end{Shaded}

\includegraphics{Project_files/figure-latex/combine_models-1.pdf}

\subsection{Confusion Matrix for each
Model}\label{confusion-matrix-for-each-model}

\begin{Shaded}
\begin{Highlighting}[]
\NormalTok{getAccuracy <-}\StringTok{ }\ControlFlowTok{function}\NormalTok{(model) \{}
  \KeywordTok{return}\NormalTok{(model}\OperatorTok{$}\NormalTok{result}\OperatorTok{$}\NormalTok{Accuracy[}\KeywordTok{length}\NormalTok{(model}\OperatorTok{$}\NormalTok{result}\OperatorTok{$}\NormalTok{Accuracy)])}
\NormalTok{\}}
\NormalTok{allModels.prediction <-}\StringTok{ }\KeywordTok{list}\NormalTok{()}
\NormalTok{setNumberPrecision <-}\StringTok{ }\ControlFlowTok{function}\NormalTok{(x, k) }\KeywordTok{trimws}\NormalTok{(}\KeywordTok{format}\NormalTok{(}\KeywordTok{round}\NormalTok{(x, k), }\DataTypeTok{nsmall=}\NormalTok{k))}

\NormalTok{printConfusion <-}\StringTok{ }\ControlFlowTok{function}\NormalTok{(currentConfusionMatrix, modelName) \{}
\NormalTok{  confusionMatrixAsDataFrame <-}\StringTok{ }\KeywordTok{data.frame}\NormalTok{(currentConfusionMatrix}\OperatorTok{$}\NormalTok{table)}
\NormalTok{  confmatrix_df <-}\StringTok{ }\KeywordTok{data.frame}\NormalTok{(currentConfusionMatrix}\OperatorTok{$}\NormalTok{table)}
\NormalTok{  plotConfSquares <-}\StringTok{ }\KeywordTok{ggplot}\NormalTok{(confmatrix_df) }\OperatorTok{+}\StringTok{ }\KeywordTok{geom_tile}\NormalTok{(}\KeywordTok{aes}\NormalTok{(}\DataTypeTok{x=}\NormalTok{Prediction, }\DataTypeTok{y=}\NormalTok{Reference, }\DataTypeTok{fill=}\NormalTok{Freq))}

  \KeywordTok{cat}\NormalTok{(}\StringTok{"### Model "}\NormalTok{,modelName,}\StringTok{"}\CharTok{\textbackslash{}n}\StringTok{"}\NormalTok{)}
\NormalTok{  currentModel <-}\StringTok{ }\NormalTok{allModels[modelName]}

\NormalTok{  label <-}\StringTok{ }\KeywordTok{paste}\NormalTok{(}\StringTok{'Confusion Matrix of model'}\NormalTok{,modelName)}
  \KeywordTok{cat}\NormalTok{(}\KeywordTok{paste0}\NormalTok{(}\StringTok{"#### "}\NormalTok{,label,}\StringTok{"}\CharTok{\textbackslash{}n\textbackslash{}n}\StringTok{"}\NormalTok{))}
  \KeywordTok{cat}\NormalTok{(}\KeywordTok{paste0}\NormalTok{(}\KeywordTok{kable}\NormalTok{(currentConfusionMatrix}\OperatorTok{$}\NormalTok{table,}\DataTypeTok{digits =} \DecValTok{4}\NormalTok{) }\OperatorTok
\StringTok{    }\KeywordTok{kable_styling}\NormalTok{(}\DataTypeTok{bootstrap_options =} \StringTok{"striped"}\NormalTok{, }\DataTypeTok{full_width =}\NormalTok{ F, }\DataTypeTok{position =} \StringTok{"center"}\NormalTok{),}\DataTypeTok{collapse=}\StringTok{"}\CharTok{\textbackslash{}n}\StringTok{"}\NormalTok{))}
  
  \KeywordTok{cat}\NormalTok{(}\KeywordTok{paste0}\NormalTok{(}\StringTok{"}\CharTok{\textbackslash{}n\textbackslash{}n\textbackslash{}n}\StringTok{#### Overall Statistics of model "}\NormalTok{,modelName,}\StringTok{"}\CharTok{\textbackslash{}n\textbackslash{}n}\StringTok{"}\NormalTok{))}
\NormalTok{  numberDigits <-}\StringTok{ }\DecValTok{4}
\NormalTok{  tableColumns <-}\StringTok{ }\KeywordTok{c}\NormalTok{(}\StringTok{"Accuracy"}\NormalTok{,}\StringTok{"95% CI"}\NormalTok{,}\StringTok{"No Information Rate"}\NormalTok{,}\StringTok{"Kappa"}\NormalTok{,}\StringTok{"Mcnemar's Test P-Value"}\NormalTok{)}
\NormalTok{  tableValues <-}\StringTok{ }\KeywordTok{c}\NormalTok{(}
    \KeywordTok{setNumberPrecision}\NormalTok{(currentConfusionMatrix}\OperatorTok{$}\NormalTok{overall[[}\StringTok{"Accuracy"}\NormalTok{]], numberDigits),       }\CommentTok{# Accuracy}
    \KeywordTok{paste0}\NormalTok{(                                                                               }\CommentTok{# Confidence Interval of accuracy}
      \StringTok{"("}\NormalTok{,}
      \KeywordTok{setNumberPrecision}\NormalTok{(currentConfusionMatrix}\OperatorTok{$}\NormalTok{overall[[}\StringTok{"AccuracyLower"}\NormalTok{]], numberDigits),}
      \StringTok{", "}\NormalTok{, }
      \KeywordTok{setNumberPrecision}\NormalTok{(currentConfusionMatrix}\OperatorTok{$}\NormalTok{overall[[}\StringTok{"AccuracyUpper"}\NormalTok{]], numberDigits),}
      \StringTok{")"}
\NormalTok{    ),}
    \KeywordTok{setNumberPrecision}\NormalTok{(currentConfusionMatrix}\OperatorTok{$}\NormalTok{overall[[}\StringTok{"AccuracyPValue"}\NormalTok{]], numberDigits), }\CommentTok{# no information rate}
    \KeywordTok{setNumberPrecision}\NormalTok{(currentConfusionMatrix}\OperatorTok{$}\NormalTok{overall[[}\StringTok{"Kappa"}\NormalTok{]], numberDigits),          }\CommentTok{# kappa}
    \KeywordTok{setNumberPrecision}\NormalTok{(currentConfusionMatrix}\OperatorTok{$}\NormalTok{overall[[}\StringTok{"McnemarPValue"}\NormalTok{]], numberDigits)   }\CommentTok{# Mcnemar's Test P-Value}
\NormalTok{  )}
\NormalTok{  tableStatistics <-}\StringTok{ }\KeywordTok{data.frame}\NormalTok{(}\DataTypeTok{statistics =}\NormalTok{ tableColumns, }\DataTypeTok{values =}\NormalTok{ tableValues)}
  \KeywordTok{cat}\NormalTok{(}\KeywordTok{paste0}\NormalTok{(}\KeywordTok{kable}\NormalTok{(tableStatistics,}\DataTypeTok{digits =} \DecValTok{4}\NormalTok{) }\OperatorTok
\StringTok{    }\KeywordTok{kable_styling}\NormalTok{(}\DataTypeTok{bootstrap_options =} \StringTok{"striped"}\NormalTok{, }\DataTypeTok{full_width =}\NormalTok{ F, }\DataTypeTok{position =} \StringTok{"center"}\NormalTok{),}\DataTypeTok{collapse=}\StringTok{"}\CharTok{\textbackslash{}n}\StringTok{"}\NormalTok{))}

\NormalTok{  label <-}\StringTok{ }\KeywordTok{paste}\NormalTok{(}\StringTok{'Statistics by Class of model'}\NormalTok{,modelName)}
  \KeywordTok{cat}\NormalTok{(}\KeywordTok{paste0}\NormalTok{(}\StringTok{"}\CharTok{\textbackslash{}n\textbackslash{}n\textbackslash{}n}\StringTok{#### "}\NormalTok{,label,}\StringTok{"}\CharTok{\textbackslash{}n\textbackslash{}n}\StringTok{"}\NormalTok{))}
  \KeywordTok{cat}\NormalTok{(}\KeywordTok{paste0}\NormalTok{(knitr}\OperatorTok{::}\KeywordTok{kable}\NormalTok{(currentConfusionMatrix}\OperatorTok{$}\NormalTok{byClass,}\DataTypeTok{digits =} \DecValTok{4}\NormalTok{) }\OperatorTok
\StringTok{    }\KeywordTok{kable_styling}\NormalTok{(}\DataTypeTok{bootstrap_options =} \StringTok{"striped"}\NormalTok{, }\DataTypeTok{full_width =}\NormalTok{ F, }\DataTypeTok{position =} \StringTok{"left"}\NormalTok{, }\DataTypeTok{font_size =} \DecValTok{11}\NormalTok{),}\DataTypeTok{collapse=}\StringTok{"}\CharTok{\textbackslash{}n}\StringTok{"}\NormalTok{))}
  \KeywordTok{cat}\NormalTok{(}\StringTok{'}\CharTok{\textbackslash{}n\textbackslash{}n}\StringTok{'}\NormalTok{)}
  
  \KeywordTok{print}\NormalTok{(plotConfSquares)}
  \KeywordTok{cat}\NormalTok{(}\StringTok{'}\CharTok{\textbackslash{}n\textbackslash{}n}\StringTok{'}\NormalTok{)}
  \ControlFlowTok{if}\NormalTok{(modelName}\OperatorTok{==}\StringTok{"rpart"}\NormalTok{)\{}
    \KeywordTok{fancyRpartPlot}\NormalTok{(currentModel}\OperatorTok{$}\NormalTok{rpart}\OperatorTok{$}\NormalTok{finalModel)}
\NormalTok{  \}  }
  \ControlFlowTok{if}\NormalTok{(modelName}\OperatorTok{==}\StringTok{"rf"}\NormalTok{)\{}
\NormalTok{    randomForestError <-}\StringTok{ }\KeywordTok{gg_error}\NormalTok{(currentModel}\OperatorTok{$}\NormalTok{rf}\OperatorTok{$}\NormalTok{finalModel)}
    \KeywordTok{print}\NormalTok{(}\KeywordTok{plot}\NormalTok{(randomForestError))}
\NormalTok{  \}}
  \KeywordTok{cat}\NormalTok{(}\StringTok{'}\CharTok{\textbackslash{}n\textbackslash{}n}\StringTok{'}\NormalTok{)}
\NormalTok{\}}

\ControlFlowTok{for}\NormalTok{(modelName }\ControlFlowTok{in} \KeywordTok{names}\NormalTok{(allModels)) \{}
  \KeywordTok{set.seed}\NormalTok{(}\DecValTok{123}\NormalTok{)}
\NormalTok{  currentModel <-}\StringTok{ }\NormalTok{allModels[modelName]}
\NormalTok{  predictedClasse <-}\StringTok{ }\KeywordTok{predict}\NormalTok{(currentModel,validationDataset)}
\NormalTok{  allModels.prediction[[modelName]] <-}\StringTok{ }\NormalTok{predictedClasse}
\NormalTok{  currentConfusionMatrix <-}\StringTok{ }\KeywordTok{confusionMatrix}\NormalTok{(predictedClasse[[modelName]], }\KeywordTok{as.factor}\NormalTok{(validationDataset}\OperatorTok{$}\NormalTok{classe))}
  \KeywordTok{printConfusion}\NormalTok{(currentConfusionMatrix, modelName)}
\NormalTok{\}}
\end{Highlighting}
\end{Shaded}

\subsubsection{Model treebag}\label{model-treebag}

\paragraph{Confusion Matrix of model
treebag}\label{confusion-matrix-of-model-treebag}

\begin{table}[H]
\centering
\begin{tabular}{l|r|r|r|r|r}
\hline
  & A & B & C & D & E\\
\hline
A & 551 & 7 & 1 & 1 & 1\\
\hline
B & 2 & 367 & 3 & 0 & 6\\
\hline
C & 0 & 2 & 334 & 11 & 6\\
\hline
D & 4 & 1 & 4 & 307 & 3\\
\hline
E & 1 & 2 & 0 & 2 & 344\\
\hline
\end{tabular}
\end{table}

\paragraph{Overall Statistics of model
treebag}\label{overall-statistics-of-model-treebag}

\begin{table}[H]
\centering
\begin{tabular}{l|l}
\hline
statistics & values\\
\hline
Accuracy & 0.9709\\
\hline
95\% CI & (0.9625, 0.9779)\\
\hline
No Information Rate & 0.0000\\
\hline
Kappa & 0.9632\\
\hline
Mcnemar's Test P-Value & 0.0510\\
\hline
\end{tabular}
\end{table}

\paragraph{Statistics by Class of model
treebag}\label{statistics-by-class-of-model-treebag}

\begingroup\fontsize{11}{13}\selectfont

\begin{tabular}{l|r|r|r|r|r|r|r|r|r|r|r}
\hline
  & Sensitivity & Specificity & Pos Pred Value & Neg Pred Value & Precision & Recall & F1 & Prevalence & Detection Rate & Detection Prevalence & Balanced Accuracy\\
\hline
Class: A & 0.9875 & 0.9929 & 0.9822 & 0.9950 & 0.9822 & 0.9875 & 0.9848 & 0.2847 & 0.2811 & 0.2862 & 0.9902\\
\hline
Class: B & 0.9683 & 0.9930 & 0.9709 & 0.9924 & 0.9709 & 0.9683 & 0.9696 & 0.1934 & 0.1872 & 0.1929 & 0.9807\\
\hline
Class: C & 0.9766 & 0.9883 & 0.9462 & 0.9950 & 0.9462 & 0.9766 & 0.9612 & 0.1745 & 0.1704 & 0.1801 & 0.9824\\
\hline
Class: D & 0.9564 & 0.9927 & 0.9624 & 0.9915 & 0.9624 & 0.9564 & 0.9594 & 0.1638 & 0.1566 & 0.1628 & 0.9745\\
\hline
Class: E & 0.9556 & 0.9969 & 0.9857 & 0.9901 & 0.9857 & 0.9556 & 0.9704 & 0.1837 & 0.1755 & 0.1781 & 0.9762\\
\hline
\end{tabular}

\endgroup{}

\includegraphics{Project_files/figure-latex/apply_model_on_validation-1.pdf}

\subsubsection{Model knn}\label{model-knn}

\paragraph{Confusion Matrix of model
knn}\label{confusion-matrix-of-model-knn}

\begin{table}[H]
\centering
\begin{tabular}{l|r|r|r|r|r}
\hline
  & A & B & C & D & E\\
\hline
A & 550 & 8 & 0 & 1 & 1\\
\hline
B & 2 & 363 & 1 & 0 & 1\\
\hline
C & 2 & 7 & 338 & 11 & 5\\
\hline
D & 3 & 0 & 3 & 307 & 2\\
\hline
E & 1 & 1 & 0 & 2 & 351\\
\hline
\end{tabular}
\end{table}

\paragraph{Overall Statistics of model
knn}\label{overall-statistics-of-model-knn}

\begin{table}[H]
\centering
\begin{tabular}{l|l}
\hline
statistics & values\\
\hline
Accuracy & 0.9740\\
\hline
95\% CI & (0.9659, 0.9806)\\
\hline
No Information Rate & 0.0000\\
\hline
Kappa & 0.9671\\
\hline
Mcnemar's Test P-Value & NaN\\
\hline
\end{tabular}
\end{table}

\paragraph{Statistics by Class of model
knn}\label{statistics-by-class-of-model-knn}

\begingroup\fontsize{11}{13}\selectfont

\begin{tabular}{l|r|r|r|r|r|r|r|r|r|r|r}
\hline
  & Sensitivity & Specificity & Pos Pred Value & Neg Pred Value & Precision & Recall & F1 & Prevalence & Detection Rate & Detection Prevalence & Balanced Accuracy\\
\hline
Class: A & 0.9857 & 0.9929 & 0.9821 & 0.9943 & 0.9821 & 0.9857 & 0.9839 & 0.2847 & 0.2806 & 0.2857 & 0.9893\\
\hline
Class: B & 0.9578 & 0.9975 & 0.9891 & 0.9900 & 0.9891 & 0.9578 & 0.9732 & 0.1934 & 0.1852 & 0.1872 & 0.9776\\
\hline
Class: C & 0.9883 & 0.9845 & 0.9311 & 0.9975 & 0.9311 & 0.9883 & 0.9589 & 0.1745 & 0.1724 & 0.1852 & 0.9864\\
\hline
Class: D & 0.9564 & 0.9951 & 0.9746 & 0.9915 & 0.9746 & 0.9564 & 0.9654 & 0.1638 & 0.1566 & 0.1607 & 0.9758\\
\hline
Class: E & 0.9750 & 0.9975 & 0.9887 & 0.9944 & 0.9887 & 0.9750 & 0.9818 & 0.1837 & 0.1791 & 0.1811 & 0.9862\\
\hline
\end{tabular}

\endgroup{}

\includegraphics{Project_files/figure-latex/apply_model_on_validation-2.pdf}

\subsubsection{Model rpart}\label{model-rpart}

\paragraph{Confusion Matrix of model
rpart}\label{confusion-matrix-of-model-rpart}

\begin{table}[H]
\centering
\begin{tabular}{l|r|r|r|r|r}
\hline
  & A & B & C & D & E\\
\hline
A & 507 & 221 & 318 & 154 & 147\\
\hline
B & 0 & 0 & 0 & 0 & 0\\
\hline
C & 0 & 0 & 0 & 0 & 0\\
\hline
D & 40 & 76 & 17 & 123 & 56\\
\hline
E & 11 & 82 & 7 & 44 & 157\\
\hline
\end{tabular}
\end{table}

\paragraph{Overall Statistics of model
rpart}\label{overall-statistics-of-model-rpart}

\begin{table}[H]
\centering
\begin{tabular}{l|l}
\hline
statistics & values\\
\hline
Accuracy & 0.4015\\
\hline
95\% CI & (0.3797, 0.4236)\\
\hline
No Information Rate & 0.0000\\
\hline
Kappa & 0.2021\\
\hline
Mcnemar's Test P-Value & NaN\\
\hline
\end{tabular}
\end{table}

\paragraph{Statistics by Class of model
rpart}\label{statistics-by-class-of-model-rpart}

\begingroup\fontsize{11}{13}\selectfont

\begin{tabular}{l|r|r|r|r|r|r|r|r|r|r|r}
\hline
  & Sensitivity & Specificity & Pos Pred Value & Neg Pred Value & Precision & Recall & F1 & Prevalence & Detection Rate & Detection Prevalence & Balanced Accuracy\\
\hline
Class: A & 0.9086 & 0.4009 & 0.3764 & 0.9168 & 0.3764 & 0.9086 & 0.5323 & 0.2847 & 0.2587 & 0.6872 & 0.6547\\
\hline
Class: B & 0.0000 & 1.0000 & NaN & 0.8066 & NA & 0.0000 & NA & 0.1934 & 0.0000 & 0.0000 & 0.5000\\
\hline
Class: C & 0.0000 & 1.0000 & NaN & 0.8255 & NA & 0.0000 & NA & 0.1745 & 0.0000 & 0.0000 & 0.5000\\
\hline
Class: D & 0.3832 & 0.8847 & 0.3942 & 0.8799 & 0.3942 & 0.3832 & 0.3886 & 0.1638 & 0.0628 & 0.1592 & 0.6339\\
\hline
Class: E & 0.4361 & 0.9100 & 0.5216 & 0.8776 & 0.5216 & 0.4361 & 0.4750 & 0.1837 & 0.0801 & 0.1536 & 0.6731\\
\hline
\end{tabular}

\endgroup{}

\includegraphics{Project_files/figure-latex/apply_model_on_validation-3.pdf}

\includegraphics{Project_files/figure-latex/apply_model_on_validation-4.pdf}

\subsubsection{Model gbm}\label{model-gbm}

\paragraph{Confusion Matrix of model
gbm}\label{confusion-matrix-of-model-gbm}

\begin{table}[H]
\centering
\begin{tabular}{l|r|r|r|r|r}
\hline
  & A & B & C & D & E\\
\hline
A & 504 & 35 & 19 & 11 & 6\\
\hline
B & 11 & 295 & 24 & 3 & 29\\
\hline
C & 20 & 29 & 282 & 35 & 25\\
\hline
D & 18 & 6 & 9 & 267 & 17\\
\hline
E & 5 & 14 & 8 & 5 & 283\\
\hline
\end{tabular}
\end{table}

\paragraph{Overall Statistics of model
gbm}\label{overall-statistics-of-model-gbm}

\begin{table}[H]
\centering
\begin{tabular}{l|l}
\hline
statistics & values\\
\hline
Accuracy & 0.8321\\
\hline
95\% CI & (0.8148, 0.8484)\\
\hline
No Information Rate & 0.0000\\
\hline
Kappa & 0.7875\\
\hline
Mcnemar's Test P-Value & 0.0000\\
\hline
\end{tabular}
\end{table}

\paragraph{Statistics by Class of model
gbm}\label{statistics-by-class-of-model-gbm}

\begingroup\fontsize{11}{13}\selectfont

\begin{tabular}{l|r|r|r|r|r|r|r|r|r|r|r}
\hline
  & Sensitivity & Specificity & Pos Pred Value & Neg Pred Value & Precision & Recall & F1 & Prevalence & Detection Rate & Detection Prevalence & Balanced Accuracy\\
\hline
Class: A & 0.9032 & 0.9494 & 0.8765 & 0.9610 & 0.8765 & 0.9032 & 0.8897 & 0.2847 & 0.2571 & 0.2934 & 0.9263\\
\hline
Class: B & 0.7784 & 0.9576 & 0.8149 & 0.9474 & 0.8149 & 0.7784 & 0.7962 & 0.1934 & 0.1505 & 0.1847 & 0.8680\\
\hline
Class: C & 0.8246 & 0.9326 & 0.7212 & 0.9618 & 0.7212 & 0.8246 & 0.7694 & 0.1745 & 0.1439 & 0.1995 & 0.8786\\
\hline
Class: D & 0.8318 & 0.9695 & 0.8423 & 0.9671 & 0.8423 & 0.8318 & 0.8370 & 0.1638 & 0.1362 & 0.1617 & 0.9006\\
\hline
Class: E & 0.7861 & 0.9800 & 0.8984 & 0.9532 & 0.8984 & 0.7861 & 0.8385 & 0.1837 & 0.1444 & 0.1607 & 0.8831\\
\hline
\end{tabular}

\endgroup{}

\includegraphics{Project_files/figure-latex/apply_model_on_validation-5.pdf}

\subsubsection{Model rf}\label{model-rf}

\paragraph{Confusion Matrix of model
rf}\label{confusion-matrix-of-model-rf}

\begin{table}[H]
\centering
\begin{tabular}{l|r|r|r|r|r}
\hline
  & A & B & C & D & E\\
\hline
A & 557 & 7 & 0 & 0 & 0\\
\hline
B & 0 & 370 & 1 & 0 & 1\\
\hline
C & 0 & 2 & 340 & 11 & 2\\
\hline
D & 1 & 0 & 1 & 309 & 1\\
\hline
E & 0 & 0 & 0 & 1 & 356\\
\hline
\end{tabular}
\end{table}

\paragraph{Overall Statistics of model
rf}\label{overall-statistics-of-model-rf}

\begin{table}[H]
\centering
\begin{tabular}{l|l}
\hline
statistics & values\\
\hline
Accuracy & 0.9857\\
\hline
95\% CI & (0.9794, 0.9905)\\
\hline
No Information Rate & 0.0000\\
\hline
Kappa & 0.9819\\
\hline
Mcnemar's Test P-Value & NaN\\
\hline
\end{tabular}
\end{table}

\paragraph{Statistics by Class of model
rf}\label{statistics-by-class-of-model-rf}

\begingroup\fontsize{11}{13}\selectfont

\begin{tabular}{l|r|r|r|r|r|r|r|r|r|r|r}
\hline
  & Sensitivity & Specificity & Pos Pred Value & Neg Pred Value & Precision & Recall & F1 & Prevalence & Detection Rate & Detection Prevalence & Balanced Accuracy\\
\hline
Class: A & 0.9982 & 0.9950 & 0.9876 & 0.9993 & 0.9876 & 0.9982 & 0.9929 & 0.2847 & 0.2842 & 0.2878 & 0.9966\\
\hline
Class: B & 0.9763 & 0.9987 & 0.9946 & 0.9943 & 0.9946 & 0.9763 & 0.9854 & 0.1934 & 0.1888 & 0.1898 & 0.9875\\
\hline
Class: C & 0.9942 & 0.9907 & 0.9577 & 0.9988 & 0.9577 & 0.9942 & 0.9756 & 0.1745 & 0.1735 & 0.1811 & 0.9924\\
\hline
Class: D & 0.9626 & 0.9982 & 0.9904 & 0.9927 & 0.9904 & 0.9626 & 0.9763 & 0.1638 & 0.1577 & 0.1592 & 0.9804\\
\hline
Class: E & 0.9889 & 0.9994 & 0.9972 & 0.9975 & 0.9972 & 0.9889 & 0.9930 & 0.1837 & 0.1816 & 0.1821 & 0.9941\\
\hline
\end{tabular}

\endgroup{}

\includegraphics{Project_files/figure-latex/apply_model_on_validation-6.pdf}

\includegraphics{Project_files/figure-latex/apply_model_on_validation-7.pdf}

\subsection{Voting Mechanism}\label{voting-mechanism}

Now, let's combine all the models that have a accuracy bigger or equal
than the minimal 80\% and make them vote using the accuracy of each
model as weight.

\begin{Shaded}
\begin{Highlighting}[]
\NormalTok{voting <-}\StringTok{ }\ControlFlowTok{function}\NormalTok{(data) \{}
  \CommentTok{# Any model with worse accuracy than this, should not be considered on the voting ( score = 0 )}
\NormalTok{  MIN_ACCURACY =}\StringTok{ }\FloatTok{0.8}
  
\NormalTok{  predictions <-}\StringTok{ }\KeywordTok{list}\NormalTok{()}
\NormalTok{  predictions.accuracy <-}\StringTok{ }\KeywordTok{list}\NormalTok{()}
\NormalTok{  sumAccuracy <-}\StringTok{ }\DecValTok{0}
  \ControlFlowTok{for}\NormalTok{(modelName }\ControlFlowTok{in} \KeywordTok{names}\NormalTok{(allModels)) \{}
    \KeywordTok{set.seed}\NormalTok{(}\DecValTok{123}\NormalTok{)}
\NormalTok{    currentModel <-}\StringTok{ }\NormalTok{allModels[modelName][[modelName]]}
\NormalTok{    predictions[[modelName]] <-}\StringTok{ }\KeywordTok{predict}\NormalTok{(currentModel,data)}
\NormalTok{    predictions.accuracy[[modelName]] <-}\StringTok{ }\KeywordTok{ifelse}\NormalTok{(}
      \KeywordTok{getAccuracy}\NormalTok{(currentModel) }\OperatorTok{>}\StringTok{ }\NormalTok{MIN_ACCURACY, }\CommentTok{# filter by accuracy over the threshold}
      \KeywordTok{getAccuracy}\NormalTok{(currentModel), }
      \DecValTok{0}
\NormalTok{    )}
\NormalTok{    sumAccuracy <-}\StringTok{ }\NormalTok{sumAccuracy }\OperatorTok{+}\StringTok{ }\NormalTok{predictions.accuracy[[modelName]]}
\NormalTok{  \}}
\NormalTok{  predictions <-}\StringTok{ }\KeywordTok{data.frame}\NormalTok{(predictions)}
\NormalTok{  predictions.accuracy <-}\StringTok{ }\KeywordTok{data.frame}\NormalTok{(predictions.accuracy)}
\NormalTok{  possibleValues <-}\StringTok{ }\KeywordTok{unique}\NormalTok{(trainData}\OperatorTok{$}\NormalTok{classe)}
\NormalTok{  scoreByValue <-}\StringTok{ }\KeywordTok{list}\NormalTok{()}
  \CommentTok{# calculate the score for each possible value}
  \ControlFlowTok{for}\NormalTok{(possibleValue }\ControlFlowTok{in}\NormalTok{ possibleValues) \{}
\NormalTok{    columnName <-}\KeywordTok{paste0}\NormalTok{(}\StringTok{"score"}\NormalTok{,possibleValue)}
    \CommentTok{# get the score only of the rows where the }
    \CommentTok{# value is the current value, sum them}
    \CommentTok{# and divide by the sumAccuracy}
\NormalTok{    scoreByValue[[columnName]] <-}\StringTok{ }\KeywordTok{rowSums}\NormalTok{(}
      \KeywordTok{t}\NormalTok{(}\KeywordTok{t}\NormalTok{(predictions }\OperatorTok{==}\StringTok{ }\NormalTok{possibleValue) }\OperatorTok{*}\StringTok{ }\KeywordTok{as.numeric}\NormalTok{(predictions.accuracy[}\DecValTok{1}\NormalTok{,]))}
\NormalTok{    ) }\OperatorTok{/}\StringTok{ }\NormalTok{sumAccuracy}
\NormalTok{  \}  }
  \CommentTok{# replace null values by zero}
\NormalTok{  votingData <-}\StringTok{ }\KeywordTok{as.data.frame}\NormalTok{(scoreByValue)}
  \ControlFlowTok{for}\NormalTok{(possibleValue }\ControlFlowTok{in}\NormalTok{ possibleValues) \{}
\NormalTok{    columnName <-}\KeywordTok{paste0}\NormalTok{(}\StringTok{"score"}\NormalTok{,possibleValue)}
\NormalTok{    votingData[ }\KeywordTok{is.na}\NormalTok{(columnName), columnName ] <-}\StringTok{ }\KeywordTok{c}\NormalTok{(}\DecValTok{0}\NormalTok{)}
\NormalTok{  \}  }
  \CommentTok{# calculate the max value by each row}
\NormalTok{  votingData}\OperatorTok{$}\NormalTok{maxScore <-}\StringTok{ }\KeywordTok{apply}\NormalTok{(votingData,}\DecValTok{1}\NormalTok{,max)}
  \CommentTok{# vote in the value with the max value}
\NormalTok{  votingData}\OperatorTok{$}\NormalTok{voted <-}\StringTok{ }\KeywordTok{c}\NormalTok{()}
  \ControlFlowTok{for}\NormalTok{(possibleValue }\ControlFlowTok{in}\NormalTok{ possibleValues) \{}
\NormalTok{    columnName <-}\KeywordTok{paste0}\NormalTok{(}\StringTok{"score"}\NormalTok{,possibleValue)}
\NormalTok{    votingData[votingData[[columnName]] }\OperatorTok{==}\StringTok{ }\NormalTok{votingData}\OperatorTok{$}\NormalTok{maxScore, }\StringTok{"voted"}\NormalTok{] <-}\StringTok{ }\NormalTok{possibleValue}
\NormalTok{  \}}
\NormalTok{  votingData}\OperatorTok{$}\NormalTok{classe <-}\StringTok{ }\NormalTok{data}\OperatorTok{$}\NormalTok{classe}
  \KeywordTok{return}\NormalTok{(votingData)}
\NormalTok{\}}
\NormalTok{votedValidation <-}\StringTok{ }\KeywordTok{voting}\NormalTok{(validationDataset)}
\NormalTok{votingConfusionMatrix <-}\StringTok{ }\KeywordTok{confusionMatrix}\NormalTok{(}\KeywordTok{as.factor}\NormalTok{(votedValidation}\OperatorTok{$}\NormalTok{voted),}\KeywordTok{as.factor}\NormalTok{(validationDataset}\OperatorTok{$}\NormalTok{classe))}
\KeywordTok{printConfusion}\NormalTok{(votingConfusionMatrix, }\StringTok{"voting"}\NormalTok{)}
\end{Highlighting}
\end{Shaded}

\subsubsection{Model voting}\label{model-voting}

\paragraph{Confusion Matrix of model
voting}\label{confusion-matrix-of-model-voting}

\begin{table}[H]
\centering
\begin{tabular}{l|r|r|r|r|r}
\hline
  & A & B & C & D & E\\
\hline
A & 557 & 8 & 0 & 0 & 0\\
\hline
B & 0 & 369 & 2 & 0 & 1\\
\hline
C & 0 & 1 & 339 & 11 & 3\\
\hline
D & 1 & 0 & 1 & 309 & 2\\
\hline
E & 0 & 1 & 0 & 1 & 354\\
\hline
\end{tabular}
\end{table}

\paragraph{Overall Statistics of model
voting}\label{overall-statistics-of-model-voting}

\begin{table}[H]
\centering
\begin{tabular}{l|l}
\hline
statistics & values\\
\hline
Accuracy & 0.9837\\
\hline
95\% CI & (0.9770, 0.9888)\\
\hline
No Information Rate & 0.0000\\
\hline
Kappa & 0.9793\\
\hline
Mcnemar's Test P-Value & NaN\\
\hline
\end{tabular}
\end{table}

\paragraph{Statistics by Class of model
voting}\label{statistics-by-class-of-model-voting}

\begingroup\fontsize{11}{13}\selectfont

\begin{tabular}{l|r|r|r|r|r|r|r|r|r|r|r}
\hline
  & Sensitivity & Specificity & Pos Pred Value & Neg Pred Value & Precision & Recall & F1 & Prevalence & Detection Rate & Detection Prevalence & Balanced Accuracy\\
\hline
Class: A & 0.9982 & 0.9943 & 0.9858 & 0.9993 & 0.9858 & 0.9982 & 0.9920 & 0.2847 & 0.2842 & 0.2883 & 0.9963\\
\hline
Class: B & 0.9736 & 0.9981 & 0.9919 & 0.9937 & 0.9919 & 0.9736 & 0.9827 & 0.1934 & 0.1883 & 0.1898 & 0.9859\\
\hline
Class: C & 0.9912 & 0.9907 & 0.9576 & 0.9981 & 0.9576 & 0.9912 & 0.9741 & 0.1745 & 0.1730 & 0.1806 & 0.9910\\
\hline
Class: D & 0.9626 & 0.9976 & 0.9872 & 0.9927 & 0.9872 & 0.9626 & 0.9748 & 0.1638 & 0.1577 & 0.1597 & 0.9801\\
\hline
Class: E & 0.9833 & 0.9988 & 0.9944 & 0.9963 & 0.9944 & 0.9833 & 0.9888 & 0.1837 & 0.1806 & 0.1816 & 0.9910\\
\hline
\end{tabular}

\endgroup{}

\includegraphics{Project_files/figure-latex/create_voting_mechanism-1.pdf}

\subsection{Choosing the final Model}\label{choosing-the-final-model}

The voting process show a good result and because it is combining
different approachs it is more hard to have the same type of
overfitting.

\subsection{Show prediction on Test
Data}\label{show-prediction-on-test-data}

\begin{Shaded}
\begin{Highlighting}[]
\NormalTok{predictSamples <-}\StringTok{ }\KeywordTok{voting}\NormalTok{(testData)}
\NormalTok{knitr}\OperatorTok{::}\KeywordTok{kable}\NormalTok{(predictSamples,}\DataTypeTok{caption=}\StringTok{'predicting classe of the test data based on the model'}\NormalTok{,}\DataTypeTok{align =} \StringTok{"c"}\NormalTok{) }\OperatorTok
\StringTok{  }\KeywordTok{kable_styling}\NormalTok{(}\DataTypeTok{bootstrap_options =} \KeywordTok{c}\NormalTok{(}\StringTok{"striped"}\NormalTok{, }\StringTok{"hover"}\NormalTok{, }\StringTok{"condensed"}\NormalTok{))}
\end{Highlighting}
\end{Shaded}

\begin{table}[t]

\caption{\label{tab:apply_on_test_data}predicting classe of the test data based on the model}
\centering
\begin{tabular}{c|c|c|c|c|c|c}
\hline
scoreA & scoreB & scoreC & scoreD & scoreE & maxScore & voted\\
\hline
0.0000000 & 0.7704635 & 0.2295365 & 0.0000000 & 0 & 0.7704635 & B\\
\hline
1.0000000 & 0.0000000 & 0.0000000 & 0.0000000 & 0 & 1.0000000 & A\\
\hline
0.4899709 & 0.2646768 & 0.2453523 & 0.0000000 & 0 & 0.4899709 & A\\
\hline
1.0000000 & 0.0000000 & 0.0000000 & 0.0000000 & 0 & 1.0000000 & A\\
\hline
1.0000000 & 0.0000000 & 0.0000000 & 0.0000000 & 0 & 1.0000000 & A\\
\hline
0.0000000 & 0.0000000 & 0.0000000 & 0.0000000 & 1 & 1.0000000 & E\\
\hline
0.0000000 & 0.0000000 & 0.0000000 & 1.0000000 & 0 & 1.0000000 & D\\
\hline
0.0000000 & 0.7704635 & 0.0000000 & 0.2295365 & 0 & 0.7704635 & B\\
\hline
1.0000000 & 0.0000000 & 0.0000000 & 0.0000000 & 0 & 1.0000000 & A\\
\hline
1.0000000 & 0.0000000 & 0.0000000 & 0.0000000 & 0 & 1.0000000 & A\\
\hline
0.4899709 & 0.5100291 & 0.0000000 & 0.0000000 & 0 & 0.5100291 & B\\
\hline
0.0000000 & 0.0000000 & 1.0000000 & 0.0000000 & 0 & 1.0000000 & C\\
\hline
0.0000000 & 1.0000000 & 0.0000000 & 0.0000000 & 0 & 1.0000000 & B\\
\hline
1.0000000 & 0.0000000 & 0.0000000 & 0.0000000 & 0 & 1.0000000 & A\\
\hline
0.0000000 & 0.0000000 & 0.0000000 & 0.0000000 & 1 & 1.0000000 & E\\
\hline
0.0000000 & 0.0000000 & 0.0000000 & 0.0000000 & 1 & 1.0000000 & E\\
\hline
1.0000000 & 0.0000000 & 0.0000000 & 0.0000000 & 0 & 1.0000000 & A\\
\hline
0.0000000 & 1.0000000 & 0.0000000 & 0.0000000 & 0 & 1.0000000 & B\\
\hline
0.0000000 & 1.0000000 & 0.0000000 & 0.0000000 & 0 & 1.0000000 & B\\
\hline
0.0000000 & 1.0000000 & 0.0000000 & 0.0000000 & 0 & 1.0000000 & B\\
\hline
\end{tabular}
\end{table}

\begin{Shaded}
\begin{Highlighting}[]
\KeywordTok{cat}\NormalTok{(predictSamples}\OperatorTok{$}\NormalTok{voted)}
\end{Highlighting}
\end{Shaded}

\begin{verbatim}
## B A A A A E D B A A B C B A E E A B B B
\end{verbatim}

\subsection{Conclusion}\label{conclusion}

The final model used is a voting from the best models created from the
data. The accuracy of the voting model on the test data was very good
but not as good as on the validation dataset, as expected.

\subsection{Biografy}\label{biografy}

{[}1{]} Velloso, E.; Bulling, A.; Gellersen, H.; Ugulino, W.; Fuks, H.
\href{http://groupware.les.inf.puc-rio.br/public/papers/2013.Velloso.QAR-WLE.pdf}{Qualitative
Activity Recognition of Weight Lifting Exercises}. Proceedings of 4th
International Conference in Cooperation with SIGCHI (Augmented Human
'13) . Stuttgart, Germany: ACM SIGCHI, 2013.


\end{document}
